\section{Results}
\subsection{Outline of Results}
In this section we describe the main results from the paper. The results are themselves presented as four independent
results, which build on previous results. The results, and their relations are as listed below:
\begin{itemize}
\item The paper shows that $\Sigma$ is the weakest failure detector required to build atomic (linearizable) registers.
\item The previous result is combined with the result from \cite{lo1994using}, and CHT~\cite{chandra1996weakest} to show
that $(\Sigma, \Omega)$ is the weakes failure detector required to solve consensus in all environments.
\item A weakened version of consensus, Quittable Consensus (QC) is then presented, and it is shown that $\Psi$ is the weakest
failure detector for solving Quittable Consensus.
\item Lastly, the  authors show that one can solve NBAC by agumenting any Quittable Consensus algorithm with the
$\mathcal{FS}$ failure detector, and prove that $(\Psi, \mathcal{FS})$ is the weakest failure detector required to solve
NBAC.
\end{itemize}

\subsection{The Weakest Failure Detector for Atomic Registers}
\textbf{Result 1} $\Sigma$ (the quorum failure detector) is the weakest failure detector for implementing atomic
registers.
\textbf{Proof:} We present some of the arguments leading to this result in both directions below.

\textbf{$\Sigma \implies$ Atomic Registers:} We know from Chapter 4 of~\cite{cachin2011introduction}
and~\cite{vitanyi1986atomic} that a single-writer, single-reader atomic register (henceforth a $(1,1)$-atomic register)
can be converted to a multi-writer, multi-reader atomic register (henceforth a $(N, N)$-atomic register). We therefore
focus on producing a $(1, 1)$ atomic register using $\Sigma$. 

Quorums (majority voting) can be used to build atomic registers (particularly $(1, 1)$-atomic registers, where the
reader must merely keep track of the last value returned to guarantee linearizability). These implementations exploit
the intersection property offered by quorums, \ie every quorum (majority) must intersect with every other quorum. Since
all writes, and reads go to, and are acknowledged by a quorum, the algorithm is guaranteed that when reading from a
quorum at least one of the members must have witnessed the last write, and is therefore guaranteed to read the last
value written. $\Sigma$ (the quorum failure detector) provides algorithms with a set of trusted processes, for which
intersection is guaranteed, without requiring that a node contact a majority. By moving the intersection requirement
into the  failure detector, $\Sigma$ provides enough information to allow all reads and writes to share at least one
common node, and therefore prvide the guarantees required by atomic registers. 
