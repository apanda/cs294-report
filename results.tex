\section{Results}
\subsection{Outline of Results}
In this section we describe the main results from the paper. The results are themselves presented as four independent
results, which build on previous results. The results, and their relations are as listed below:
\begin{itemize}
\item The paper shows that $\Sigma$ is the weakest failure detector required to build atomic (linearizable) registers.
\item The previous result is combined with the result from \cite{lo1994using}, and CHT~\cite{chandra1996weakest} to show
that $(\Sigma, \Omega)$ is the weakest failure detector required to solve consensus in all environments.
\item A weakened version of consensus, Quittable Consensus (QC) is then presented, and it is shown that $\Psi$ is the weakest
failure detector for solving Quittable Consensus.
\item Lastly, the  authors show that one can solve NBAC by augmenting any Quittable Consensus algorithm with the
$\mathcal{FS}$ failure detector, and prove that $(\Psi, \mathcal{FS})$ is the weakest failure detector required to solve
NBAC.
\end{itemize}

\subsection{The Weakest Failure Detector for Atomic Registers}
\textbf{Result 1} $\Sigma$ (the quorum failure detector) is the weakest failure detector for implementing atomic
registers.
\textbf{Proof:} We present some of the arguments leading to this result in both directions below.

\textbf{$\Sigma \implies$ Atomic Registers:} We know from Chapter 4 of~\cite{cachin2011introduction}
and~\cite{vitanyi1986atomic} that a single-writer, single-reader atomic register (henceforth a $(1,1)$-atomic register)
can be converted to a multi-writer, multi-reader atomic register (henceforth a $(N, N)$-atomic register). We therefore
focus on implementing $(1, 1)$-atomic registers using $\Sigma$. $(1, 1)$-atomic registers provide the following properties:
\begin{itemize}
\item \textbf{Termination}: If a node is correct, each read and write operation eventually completes.
\item \textbf{Validity}: A read that is not concurrent with a write returns the last value written. In case of a
concurrent write, a read can return either the last value written, or the value of the concurrent write. 
\item \textbf{Ordering}: If read $r_1$ precedes read $r_2$, then $write(r_1) \leq write(r_2)$ (\ie the write for $r_2$
cannot precede the write for $r_1$.
\end{itemize}

Quorums (accessing a majority) can be used to build $(1, 1)$-atomic registers. The crucial property provided by quorums is
one of intersection, \ie any two quorums must intersect. For this result, we observe that processes trusted by $\Sigma$
exhibit the same intersection property and make use of this observation. The algorithm is listed in
Algorithm~\ref{alg:swsr}.
Briefly we discuss why all the required conditions are satisfied by this algorithm:
\begin{itemize}
\item \textbf{Termination} Since $\Sigma$ eventually only trusts correct processes, both the read and write must
eventually receive messages from all trusted processes, thus guaranteeing termination.
\item \textbf{Validity} By the intersection property of $\Sigma$, the set of processes acknowledging a write, must
intersect with the set of processes participating in a read. Since the read function picks the highest sequence number
returned, it will always return the last written value, or a value being written concurrently.
\item \textbf{Ordering} The update in line 28 and 29 ensure the ordering property is met.
\end{itemize}

\begin{algorithm}
\caption{Algorithm for $\Sigma \implies$ atomic register}
\label{alg:swsr}
\begin{algorithmic}[1]
\Procedure{Initialize}{ }
    \LineComment{Run at all processes}
    \State $current \gets \bot$
    \State $seq \gets -1$
    \Upon{$(WRITE, y, s)$}{the writer}
        \If $s > seq$
           \State $current \gets y$
           \State $seq \gets s$
           \State send(WACK) to writer
        \EndIf
    \EndUpon
    \Upon{$(READ)$}{the reader}
        \State send((RACK, seq, current)) to reader
    \EndUpon
\EndProcedure
\Procedure {write}{$x$}
\LineComment{Run at writer}
    \State $current \gets x$
    \State $seq \gets seqs + 1$
    \State send(WRITE, current, seq)
    \State wait until recv WACK from all processes trusted by $\Sigma$
\EndProcedure
\Procedure {read}{ }
    \State send(READ) to all
    \State wait until receive (RACK, s, current) from all processes trusted by $\Sigma$
    \State $a \gets max\{v | (RACK, s, *)$ is a receive message $\}$
    \If{$a > seq$}
        \State $seq \gets a$
        \State $current \gets v$ such that $(RACK, a, v)$ received
    \EndIf
    \State return current
\EndProcedure
\end{algorithmic}
\end{algorithm}
\textbf{Atomic Registers $\implies \Sigma$}
This is a significantly more involved result, so we only go over a few of the lemmas used to prove this part of the
result, and a brief sketch of the proof. The full proof is presented in~\cite{delporte2003shared}.  Overall, we need to find an algorithm to emulate $\Sigma$ using atomic
registers. For this algorithm we assume the use of $n$ (the number of processes) single-writer, multiple-reader
registers. Each process is assigned a register, we denote the registers assigned to process $p_i$ as $reg_i$.
The algorithm proceeds as following:
\begin{itemize}
\item All processes repeatedly write a value described later to their registers. All processes participating
in a write are tagged (\ie any processes that send or receive messages related to writes are tracked and the writer
learns of these processes). Associate with each write $k$ to register $reg_i$, a value $P_i(k)$, the set of all
processes (including the writer) that participated in a write. Assume $P_i(0) = \Pi$ (the first write is $k = 1$). For the points below
consider the $k^{th}$ write to $reg_i$.
\item The value written to register $reg_i$  during the $k^{th}$ write is the list $\{P_i(0), P_i(1), \ldots P_i(k -
1)\}$, \ie the list of all sets of processes that have participated in a write.
\item Each process reads all $n$ registers, and waits to hear from at least one process in each $P_j(t)$ where $p_j \in
\Pi$, and $P_j(t)$ has already been written to $reg_j$ (\ie a process sends messages and waits from acknowledgement from
at least one process in each set of processes that participated in a write). Let's represent by $O_i(k)$ the set of
processes from which $p_i$ hears back. The process returns the union of $P_i(k) \cup O_i(k)$ as the output of $\Sigma$,
\ie it returns the union of all processes that participated in the writing of the register, and at least one process
from processes that previously participated in any write to any register.
\end{itemize}

Keeping the above description in mind we need to show that the following properties are met:
\begin{itemize}
\item \textbf{Strong Completeness} Eventually only correct processes are trusted. 
\item \textbf{Intersection} Any two sets of values returned by the algorithm above has a non-empty intersection.
\end{itemize}

\textbf{Strong Completeness} First we look at strong completeness. Observe that at each step all the nodes trusted by the
previous algorithm have either participated in a write, or acknowledge a message from the process. Since we assume a
fail-stop model, failed processes must eventually neither send nor receive any messages, and hence assuming that the
output is continuously updated, eventually all trusted processes must be correct. For strong completeness we must
therefore only prove that the output is continuously updated. From the previous description, this requires that at least
one process that participates in any writes to a register must be correct for the entire execution.

\textbf{Lemma 1} At least one correct process (\ie a process that does not fail during the entire execution) must
participate in each write to a register.

\textbf{Proof} Assume this is not true. Then partition all the processes that participate in a write from the rest of
the processes, and allow no messages to pass between partitions. In this case the state of the distributed system where
all processes participating in a write fail before the write begins is indistinguishable from one where they fail after
the write succeeds (since no messages cross the partition). Now consider a process that did not participate in the write
that reads from the register after the write has ended. While, from the state observable by this process,  the distributed 
system is identical in both the case where the write never takes place, and the case where the write occurs, the value
read from the register differ. This is a contradiction, and therefore at least one correct process must have
participated in the write to the register.

From Lemma 1, we know that at least one of the processes in each set $P_i(t)$ is correct for the duration of the
execution, and therefore must eventually acknowledge messages. Therefore, the value returned by the algorithm satisfy
strong completeness.

\textbf{Intersection} Since the value returned by the algorithm has at least one process that has participated in all
previous register writes (and hence was the result of $\Sigma(p, t)$ for some process $p$ and time $t$), the
intersection property is trivially satisfied.

Since $\Sigma$ can be used to implement atomic registers, and atomic registers (and hence any failure detector that can
be used to implement them) can be used to implement $\Sigma$, we find that $\Sigma$ is the weakest failure detector for
building atomic registers.

\subsection{The Weakest Failure Detector to Solve Consensus} 
\textbf{Result 2} $(\Sigma, \Omega)$ is the weakest failure detector to solve consensus in any environment (\ie with any
number of failing processes).

\textbf{Proof}\\
\textbf{$(\Sigma, \Omega) \implies$ consensus} From the previous result we know that $\Sigma$ is the weakest failure
detector to implement atomic registers. The result from~\cite{lo1994using} shows that one can implement consensus using
shared registers and $\Omega$ in any environment. The algorithm proposed in that paper relies on $n$ $(1,N)$-atomic
registers (one per-process), and is very similar to Hierarchical Consensus with two modifications: a) Instead of using
best-effort broadcast, nodes record their decisions in the registers, and b) nodes change rounds either when the
register for the coordinator has changed, or the coordinator is suspected by the failure detector. Since the algorithm
assumes that the atomic registers themselves are fault tolerant, and can be read even after the writing processes have
died, this modified hierarchical consensus can be run with $\Omega$ and achieve consensus even in environments where all
but one of the processes fail.

\textbf{Consensus $\implies (\Sigma, \Omega)$} This result is presented in two parts:
\begin{itemize}
\item \textbf{Consensus $\implies \Omega$} The valence proof from CHT~\cite{chandra1996weakest} shows that one can
extract $\Omega$ from any implementation of the consensus algorithm. We therefore know that given consensus one can
implement $\Omega$, and therefore any failure detector which can be used to implement consensus is reducible to
$\Omega$.
\item \textbf{Consensus $\implies \Sigma$} Consensus can be used to implement atomic registers (since consensus can be
used to decide on the ordering of writes, etc). We have shown above that one can use atomic registers to implement
$\Sigma$. The combination of these two statements means that one can reduce any failure detector that can be used to
solve consensus to $\Sigma$.
\end{itemize}
As an aside, the previous result indicates that for environments where $\Omega$ is sufficient to implement consensus
(\ie environments where at least a majority of the processes are correct), $\Omega$ is reducible to $\Sigma$. Other
results in~\cite{delporte2003shared} show that a) for $2$ processes, $\Sigma$ and $\Omega$ are equivalent. Furthermore
in an environment, with $3$ or more processes $(\Sigma, \Omega)$ is strictly stronger than $\Sigma$, and is stronger
than $\Omega$ in all environments where more than a majority of processes can fail.

\subsection{The Weakest Failure Detector to Solve Quittable Consensus}
Quittable Consensus (QC) is a weakened form of consensus, where in addition to a proposed value, all processes can decide on
a special value $QUIT$ indicating that one or more failures was detected. We specifically focus on binary uniform
consensus here, in which case the values that can be decided upon are one of $\{0, 1, QUIT\}$. An algorithm for binary,
uniform Quittable Consensus must exhibit the following three properties:
\begin{itemize}
\item \textbf{Termination} If every correct process proposes a value, then every cor rec process eventual decides.
\item \textbf{Uniform Agreement} No two processes decide on different values.
\item \textbf{Validity} A process may only decide a value $v\in \{0, 1, QUIT\}$. Furthermore:
\begin{itemize}
\item If $v\in {0, 1}$ then some process previously proposed $v$.
\item If $v = QUIT$ then a failure previously occurred.
\end{itemize}
\end{itemize}
\textbf{Result 3} $\Psi$ is the weakest failure detector for solving Quittable Consensus.

\textbf{Proof}

\textbf{$\Psi \implies$ QC} This follows from the observation that $\Psi$ eventually either returns $RED$ indicating
that a failure has cured, or returns values from the range of $(\Sigma, \Omega)$. The algorithm then works as follows:
\begin{itemize}
\item Each process waits until $\Psi$ returns something other than $\bot$.
\item If $\Psi$ returns something with range $\{RED, GREEN\}$, we know by definition a failure must have been detected.
The process therefore decides $QUIT$ (since $\Psi$ presents the same behavior at all processes, we know that every other
process must also decide $QUIT$ thus meeting the Uniform Agreement condition).
\item If $\Psi$ behaves like $(\Sigma, \Omega)$ (which it will do at all processes, for the rest of the execution), use
the algorithm from result $2$ to implement consensus.
\end{itemize}

\textbf{QC $\implies \Psi$} This result relies on a proof similar to CHT~\cite{chandra1996weakest}. The algorithm relies
on building increasingly longer DAGs of failure detector readings from various processes at various times. It then
builds a simulation forest by simulating runs of the QC algorithm with different initial proposals. If the simulation in
any of the DAGs decide $QUIT$, the algorithm simulates runs a round of QC with the proposal $0$. On the other hand if
this is not the case, the algorithm runs a round of QC and proposes a value that encodes in it two initial conditions
which differ in exactly one node and schedules that result in the processes deciding different values. Using this
information, the algorithm can extract the critical index like in CHT to generate $\Omega$, and can also simulate
$\Sigma$. The full proof is presented in~\cite{guerraoui2002weakest}.

\subsection{The Weakest Failure Detector to Solve NBAC}
\textbf{Result 4} For all environments $(\Psi, \mathcal{FS})$ is the weakest failure detector to implement NBAC. 

\textbf{Proof}\\
\textbf{$(\Psi, \mathcal{FS}) \implies$ NBAC} We have shown above that $\Psi$ is the weakest failure detector required
to implement QC. In Algorithm~\ref{alg:qctonbac} we provide an algorithm for implementing NBAC using $QC$ and the
$\mathcal{FS}$ failure detector.

\begin{algorithm}
\caption{Algorithm for QC + $\mathcal{FS} \implies$ NBAC}
\label{alg:qctonbac}
\begin{algorithmic}[1]
\LineComment{Run at all processes}
\Procedure{Vote}{$v$}\Comment{$v \in \{Yes, No\}$}
    \State send v to all processes
    \State wait until receiving vote from each process or $\mathcal{FS} = RED$
    \If all processes vote yes
        \State $myproposal \gets 1$
    \Else \Comment{Some process voted no, or a failure was detected}
        \State $myproposal \gets 0$
    \EndIf
    \State $decision \gets QC(myproposal)$
    \If $decision = 1$
        \State return Commit
    \Else \Comment{$decision \in \{0, QUIT\}$}
        \State return Abort
    \EndIf
\EndProcedure
\end{algorithmic}
\end{algorithm}

We now show that Algorithm~\ref{alg:qctonback} meets all the requirements for NBAC:
\begin{itemize}
\item \textbf{Termination} If every correct process votes, then every correct process eventually returns a value.

This is true from the fact that the process waits until either votes from all processes are received, or the failure
detector indicates a failure. In either case, the algorithm must eventually make progress.

\item \textbf{Uniform Agreement} No two processes (whether correct or faulty) return different values.

True from the properties of $QC$, and how the algorithm decides.

\item \textbf{Validity} A process may only return Commit or Abort. Furthermore if the process returns $Commit$ then all
processes previously voted $Yes$. If the process returns $Abort$ either some process previously voted $no$ or a failure
previously occurred.
\end{itemize}

The algorithm only returns $Commit$ or $Abort$. If a process proposes $1$ it has seen a vote of $Yes$ from every other
process, implying every process has voted $1$. Therefore if the processes return $Commit$, at least one process
observed a $Yes$ vote from every process in the system. To propose $0$ either one of the processes voted $No$, or a
failure was detected, and hence in cases where no process votes $No$, and no process is faulty $0$ is never proposed. By
validity of QC, if QC returns $0$ or Quit, either: a) a process voted $No$, or b) a process failed.

\textbf{NBAC $\implies (\Psi, \mathcal{FS})$} Here we show algorithms for using NBAC to implement $\mathcal{FS}$ and quittable
consensus. Since from Result 3 any algorithm for Quittable Consensus can be used to implement $\Psi$, this result shows
that any NBAC implementation can be used to implement $(\Psi, \mathcal{FS})$.

\begin{algorithm}
\caption{Algorithm for NBAC $\implies \mathcal{FS}$}
\label{alg:nbactofs}
\begin{algorithmic}[1]
\LineComment{Run at all processes}
\Procedure{FS}{f}
    \State $Out \gets Green$ \Comment{$Out$ contains the output of $\mathcal{FS}$}
    \While{$Out = GREEN$} \Comment{Run forever}
        \State $result \gets NBAC.Vote(Yes)$
        \If $result = Abort$
            \State $Out \gets RED$
        \EndIf
    \EndWhile
\EndProcedure
\end{algorithmic}
\end{algorithm}

\begin{algorithm}
\caption{Algorithm for NBAC $\implies$ QC}
\label{alg:nbactoqc}
\begin{algorithmic}[1]
\LineComment{Run at all processes}
\Procedure{Propose}{$v$}\Comment{$v\in \{0, 1\}$}
    \State send v to all processes
    \State $d \gets NBAC.Vote(Yes)$
    \If $d = Abort$
        \State return Quit
    \Else
        \State wait until receiving proposal from each process
        \State return smallest proposal received
    \EndIf
\EndProcedure
\end{algorithmic}
\end{algorithm}

Algorithm~\ref{alg:nbactofs} shows an algorithm that implements $\mathcal{FS}$ using NBAC. Since each process always
votes $Yes$, the only case in which the NBAC algorithm returns $Abort$ is when a process fails. At this point the output
changes to $RED$ and stays $RED$. By the Termination property of NBAC we know that while $Out$ is $GREEN$, the
variable $Out$ is going to be eventually updated, and hence this algorithm satisfies the requirements for the
$\mathcal{FS}$ failure detector.

Algorithm~\ref{alg:nbactoqc} presents an algorithm to convert NBAC to QC. We assume the use of reliable links, and
hence if a process votes $YES$, we are guaranteed to receive a proposal from that process. We therefore meet the
Termination requirement for QC, since eventually either the NBAC module aborts (by the NBAC termination condition) or we
receive a vote from everyone. Since we pick the smallest received vote, and everyone waits for all votes to appear, we
trivially meet the Uniform Agreement property. By a similar argument this algorithm also meets the Validity conditions.

We have therefore shown that $(\Psi, \mathcal{FS})$ is the weakest failure detector required to solve NBAC.
